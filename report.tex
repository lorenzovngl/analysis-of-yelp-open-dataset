\documentclass[12pt]{article}

\usepackage[utf8]{inputenc}
\usepackage[italian]{babel}

\usepackage{biblatex}
\addbibresource{references.bib}

\title{Analisi di Yelp Open Dataset}
\author{
        Lorenzo Vainigli\\
        \textit{\small Corso di Intelligenza Artificiale a.a. 2019/20}\\
        \textit{\small Laurea Magistrale in Informatica}\\
        \textit{\small Università di Bologna}
}
\date{}

\begin{document}
\maketitle

\section{Introduzione}

\section{Dati}
\textit{Yelp Open Dataset} \cite{yelp} è una base di dati che raccoglie informazioni su esercizi commerciali di varie categorie. I dati sono utilizzabili per uso personale, educativo o accademico, sono disponibili in formato JSON e sono divisi in alcuni file:
\begin{itemize}
\item \textit{business.json}: contiene le informazioni relative agli esercizi commerciali tra cui ubicazione e categoria.
\item \textit{review.json}: contiene i testi delle recensioni includendo l'identificativo dell'utente che ha scritto la recensione e l'esercizio commerciale oggetto della recensione.
\item \textit{user.json}: contiene i dati associati ai singoli utenti, inclusi gli identificativi degli amici.
\item \textit{tip.json}: contiene dei suggerimenti che gli utenti scrivono a proposito degli esercizi commerciali. Possono essere visti come delle brevissime recensioni.
\end{itemize}
Il database contiene anche i file \textit{checkin.json} e \textit{photos.json}, ma non sono stati presi in considerazione per lo sviluppo di questo progetto.

\section{Obiettivi}

\section{Sviluppo}

\section{Risultati}

\section{Conclusioni}

\printbibliography[title={Riferimenti}]

\end{document}