\documentclass[12pt]{article}

\usepackage[utf8]{inputenc}
\usepackage[italian]{babel}
\usepackage{enumitem}

\usepackage[backend=bibtex,sorting=none]{biblatex}
\addbibresource{references.bib}

\title{Analisi di Yelp Open Dataset}
\author{
        Lorenzo Vainigli\\
        \textit{\small Corso di Intelligenza Artificiale a.a. 2019/20}\\
        \textit{\small Laurea Magistrale in Informatica}\\
        \textit{\small Università di Bologna}
}
\date{}

\begin{document}
\maketitle

\section{Introduzione}
I file di questo progetto sono disponibili nel repository dell'autore su GitHub \cite{repo}. 

\section{Dati}
\textit{Yelp Open Dataset} \cite{yelp} è una base di dati che raccoglie informazioni su esercizi commerciali di varie categorie. I dati sono utilizzabili per uso personale, educativo o accademico, sono disponibili in formato JSON e sono divisi in alcuni file:
\begin{itemize}
\item \textit{business.json}: contiene le informazioni relative agli esercizi commerciali tra cui ubicazione e categoria.
\item \textit{review.json}: contiene i testi delle recensioni includendo l'identificativo dell'utente che ha scritto la recensione e l'esercizio commerciale oggetto della recensione.
\item \textit{user.json}: contiene i dati associati ai singoli utenti, inclusi gli identificativi degli amici.
\item \textit{tip.json}: contiene dei suggerimenti che gli utenti scrivono a proposito degli esercizi commerciali. Possono essere visti come delle brevissime recensioni.
\end{itemize}
Il database contiene anche i file \textit{checkin.json} e \textit{photos.json}, ma non sono stati presi in considerazione per lo sviluppo di questo progetto.

\section{Obiettivi}
Lo scopo del progetto prevede l'analisi dei dati al fine di studiare la loro struttura e il loro contenuto, al fine di estrapolare osservazioni interessanti su di essi. Non si tratta solo di aggregare record o trovare valori minimi, massimi o medi, ma di applicare anche tecniche di NLP e machine learning. In partcolare, le finalità del progetto richiedono:
\begin{enumerate}[label=T\arabic*)]
\item il riconoscimento automatico di una review positiva o negativa;
\item il raggruppamento degli utenti in base alle loro preferenze o compor-
tamento sulla piattaforma;
\item il raggruppamento automatico dei locali in base a criteri di similitudine data una certa località.
\end{enumerate}
A queste, ne sono state aggiunte altre:
\begin{enumerate}[label=T\arabic*)]
\setcounter{enumi}{3}
\item analisi dei singoli file JSON;
\item classificazione dei locali migliori e peggiori per ogni categoria;
\item locali aperti nelle vicinanze dell'utente;
\item utenti con le recensioni più affidabili (comparando il loro voto alla media dei voti di un determinato locale);
\item migliori recensioni e suggerimenti (tips) per un locale.
\end{enumerate}

\section{Strumenti}
Per conseguire gli obiettivi sopra citati i dati sono stati elaborati in Python con l'utilizzo di Jupyter Notebook \cite{jupyter}.

\section{Sviluppo}
Per ogni obiettivo (o target) T* è stato creato un notebook presente nella cartella \texttt{notebooks}:
\begin{enumerate}[label=T\arabic*)]
\item \texttt{reviews\_classification.ipynb};
\item \texttt{users\_grouping.ipynb};
\item \texttt{businesses\_grouping.ipynb};
\item \texttt{business.ipynb}, \texttt{review.ipynb}, \texttt{tip.ipynb}, \texttt{user.ipynb};
\item \texttt{best\_and\_worst\_businesses.ipynb};
\item \texttt{closest\_opened\_businesses.ipynb};
\item \texttt{best\_reviewers.ipynb};
\item \texttt{best\_business\_tips.ipynb}.
\end{enumerate}

\section{Risultati}

\section{Conclusioni}

\printbibliography[title={Riferimenti}]

\end{document}